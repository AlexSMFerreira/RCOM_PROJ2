\chapter{Part 2 -- Network Configuration and Analysis}
% Repeat for each experiment (1 to 6)
\section{Experiment 1 - Configure an IP Network}
\subsection{Network Architecture}
For this experiment, we connected TUX3 and TUX4 through the switch and 
configured their IP addresses as requested in the project description.
\subsection{Objectives}
State the learning objectives.
\subsection{Main Configuration Commands}
List the main commands/scripts used.
\subsection{Relevant Logs}
Show relevant logs and outputs.
\subsection{Analysis}
Discuss the results and learning points.

\section{Experiment 2 - Implement two bridges in a switch}
\subsection{Network Architecture}
Describe the network setup for this experiment.
\subsection{Objectives}
State the learning objectives.
\subsection{Main Configuration Commands}
List the main commands/scripts used.
\subsection{Relevant Logs}
Show relevant logs and outputs.
\subsection{Analysis}
Discuss the results and learning points.

\section{Experiment 3 - Configure a Router in Linux}
\subsection{Network Architecture}
Describe the network setup for this experiment.
\subsection{Objectives}
State the learning objectives.
\subsection{Main Configuration Commands}
List the main commands/scripts used.
\subsection{Relevant Logs}
Show relevant logs and outputs.
\subsection{Analysis}
Discuss the results and learning points.

\section{Experiment 4 - Configure a Commercial Router and Implement \ac{NAT}}
\subsection{Configuring a Static Route in a Commercial Router}
To configure a static route in the commercial router, we accessed its 
console and entered the necessary commands. We added a new static route with the desti\ac{NAT}ion network, subnet mask, and gateway as per the project requirements. After saving the configuration, we verified the route was correctly added by checking the routing table in the router's interface.
\subsection{\ac{ICMP} Redirection}
To test \ac{ICMP} redirection, we initiated ping requests from TUX2 to TUX3.
Initially, the packets were routed through TUX4 as it was the shortest path to subnet 172.16.Y0.0/24. After that, as requested in the project description, we disabled redirection acceptance on TUX2, and changed the routes to force the packets to go through the commercial router. This way, we observed that TUX2 continued to send packets through TUX4, ignoring the \ac{ICMP} redirect messages from the commercial router.
With \ac{ICMP} redirection disabled, after the first \ac{ICMP} redirect, TUX2 switched to its original routing path, demonstrating the effect of disabling \ac{ICMP} redirect acceptance on the routing behavior of the host.
\subsection{Configuring \ac{NAT}}
As \ac{NAT} was already enabled by default on the commercial router, to understand its functionality, we disabled it through the router's console and observed its effects on the network communication. We then re-enabled \ac{NAT} to restore normal operation.
\subsection{Network Address Translation}
To understand \ac{NAT} functionality, we performed ping tests from TUX3 to the FTP server before and after disabling \ac{NAT} on the commercial router. With \ac{NAT} enabled, TUX3 was able to successfully ping the server, as the router correctly translated the private \ac{IP} address of TUX3 to a public \ac{IP} address for communication. With \ac{NAT} disabled, the pings failed, indicating that the server could not reach TUX3's private \ac{IP} address directly. This demonstrated the importance of \ac{NAT} in allowing devices within a private network to communicate with external networks.


\section{Experiment 5 - DNS}
\subsection{Network Architecture}
Describe the network setup for this experiment.
\subsection{Objectives}
State the learning objectives.
\subsection{Main Configuration Commands}
List the main commands/scripts used.
\subsection{Relevant Logs}
Show relevant logs and outputs.
\subsection{Analysis}
Discuss the results and learning points.

\section{Experiment 6 - TCP Connections}
\subsection{Network Architecture}
Describe the network setup for this experiment.
\subsection{Objectives}
State the learning objectives.
\subsection{Main Configuration Commands}
List the main commands/scripts used.
\subsection{Relevant Logs}
Show relevant logs and outputs.
\subsection{Analysis}
Discuss the results and learning points.

