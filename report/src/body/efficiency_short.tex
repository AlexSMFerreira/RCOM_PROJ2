% Short 3-page efficiency-only report
\chapter*{Efficiency Report}
\addcontentsline{toc}{chapter}{Efficiency Report}

\section*{Summary}
This concise report presents the efficiency analysis of the implemented Data Link Layer protocol.
It focuses exclusively on metrics and methodology used to evaluate protocol performance.

\section*{Introduction}
\begin{sloppypar}
The implemented protocol follows a Stop-and-Wait \ac{ARQ} scheme with alternating sequence numbers (0/1), header (\ac{BCC}1) and data (\ac{BCC}2) checks, byte stuffing for transparency, and timeout-triggered retransmissions.
This report evaluates the protocol's efficiency under different operating conditions and parameter choices.
\end{sloppypar}

\section*{Methodology}
\subsection*{Experimental Setup}
\begin{itemize}
  \setlength{\itemsep}{0pt}
  \setlength{\parskip}{0pt}
  \setlength{\topsep}{0pt}
  \item Link: serial emulation using the provided virtual cable.
  \item Baudrate: configurable; typical tests at 38400\,bps.
  \item Test file: binary payload (example: \texttt{penguin.gif}) of varied size.
\end{itemize}

\subsection*{Metrics}
The primary metric is protocol efficiency $S$, defined as:
\[ S = \frac{\text{useful bytes delivered}}{\text{total bytes transmitted (including overhead and retransmissions)}}. \]
Secondary metrics: throughput (bytes/s), retransmission rate, and transfer time.

\subsection*{Procedure}
For each parameter set (frame size $L$, baud rate $C$, \ac{BER}, propagation delay $T_{\text{Prop}}$) record total frames transmitted, retransmissions, timeouts and elapsed time.

\section*{Results}
The results are presented below.

\subsection*{Theoretical Efficiency}
For a Stop-and-Wait \ac{ARQ} protocol, the theoretical efficiency depends on the \ac{FER}, frame size, baud rate and propagation time:
\[ S_{\text{Theory}} = \frac{(1-\text{BER})^{L}}{1 + 2(\frac{C \cdot T_{\text{Prop}}}{L})} \]
where $L$ is the frame size (bits), $BER$ is the bit error rate (the probability that a given bit is received in error), $C$ is the channel capacity (bps), and $T_{\text{Prop}}$ is the propagation delay (seconds).

\subsection*{Experimental Results}
\begin{table}[htbp]
\centering
\caption{Experimental Results Summary}
\label{tab:results}
\begin{threeparttable}
\resizebox{\textwidth}{!}{%
\begin{tabular}{c|c|c|c|c|c|c|c|c|c|c|c|c|c}
\hline
\textbf{Test} & \textbf{Baud Rate} & \textbf{BER} & \textbf{FER} & \textbf{$T_{Prop}$} & \textbf{Frame Size} & \textbf{Time} & \textbf{TX} & \textbf{ReTX} & \textbf{REJ} & \textbf{TO} & \textbf{R} & \textbf{S} & \textbf{$S_{Theory}$} \\
\textbf{Num.} & \textbf{(bps)} & & & \textbf{(ms)} & \textbf{(bytes)} & \textbf{(s)} & & & & & \textbf{(bps)} & \textbf{(\%)} & \textbf{(\%)} \\
\hline
\hline
\multicolumn{14}{c}{\textit{Test Group 1: Varying Baud Rate (BER=0, FER=0\%, $T_{Prop}$=0ms, Frame Size=900 bytes)}} \\
\hline
11 & 9600 & 0.000000 & 0.00 & 0.0 & 900 & 11.772 & 15 & 0 & 0 & 0 & 7453.62 & 77.64 & 100.00 \\
12 & 19200 & 0.000000 & 0.00 & 0.0 & 900 & 5.886 & 15 & 0 & 0 & 0 & 14907.24 & 77.64 & 100.00 \\
13 & 38400 & 0.000000 & 0.00 & 0.0 & 900 & 2.943 & 15 & 0 & 0 & 0 & 29814.48 & 77.64 & 100.00 \\
14 & 57600 & 0.000000 & 0.00 & 0.0 & 900 & 1.963 & 15 & 0 & 0 & 0 & 44698.93 & 77.60 & 100.00 \\
15 & 115200 & 0.000000 & 0.00 & 0.0 & 900 & 0.983 & 15 & 0 & 0 & 0 & 89261.44 & 77.48 & 100.00 \\
\hline
\multicolumn{14}{c}{\textit{Test Group 2: Varying BER (Baud Rate=38400, $T_{Prop}$=0ms, Frame Size=900 bytes)}} \\
\hline
21 & 38400 & 0.000000 & 0.00 & 0.0 & 900 & 3.140 & 15 & 0 & 0 & 0 & 27943.95 & 72.77 & 100.00 \\
22 & 38400 & 0.000001 & 0.72 & 0.0 & 900 & 3.146 & 15 & 0 & 0 & 0 & 27890.65 & 72.63 & 99.99 \\
23 & 38400 & 0.000003 & 2.16 & 0.0 & 900 & 3.149 & 15 & 0 & 0 & 0 & 27864.08 & 72.56 & 99.98 \\
24 & 38400 & 0.000010 & 7.20 & 0.0 & 900 & 3.647 & 17 & 2 & 2 & 0 & 24059.23 & 62.65 & 99.93 \\
25 & 38400 & 0.000030 & 21.60 & 0.0 & 900 & 4.600 & 21 & 6 & 6 & 0 & 19074.78 & 49.67 & 99.78 \\
\hline
\multicolumn{14}{c}{\textit{Test Group 3: Varying Propagation Delay (Baud Rate=38400, BER=0, Frame Size=900 bytes)}} \\
\hline
31 & 38400 & 0.000000 & 0.00 & 0.0 & 900 & 2.943 & 15 & 0 & 0 & 0 & 29814.48 & 77.64 & 100.00 \\
32 & 38400 & 0.000000 & 0.00 & 5.0 & 900 & 3.112 & 15 & 0 & 0 & 0 & 28195.37 & 73.43 & 94.94 \\
33 & 38400 & 0.000000 & 0.00 & 10.0 & 900 & 3.280 & 15 & 0 & 0 & 0 & 26751.22 & 69.66 & 90.36 \\
34 & 38400 & 0.000000 & 0.00 & 20.0 & 900 & 3.627 & 15 & 0 & 0 & 0 & 24191.89 & 63.00 & 82.42 \\
35 & 38400 & 0.000000 & 0.00 & 50.0 & 900 & 4.645 & 15 & 0 & 0 & 0 & 18889.99 & 49.19 & 65.22 \\
\hline
\multicolumn{14}{c}{\textit{Test Group 4: Varying Frame Size (Baud Rate=38400, BER=0, $T_{Prop}$=0ms)}} \\
\hline
41 & 38400 & 0.000000 & 0.00 & 0.0 & 32 & 4.680 & 345 & 0 & 0 & 0 & 18748.72 & 48.82 & 100.00 \\
42 & 38400 & 0.000000 & 0.00 & 0.0 & 64 & 3.781 & 174 & 0 & 0 & 0 & 23206.56 & 60.43 & 100.00 \\
43 & 38400 & 0.000000 & 0.00 & 0.0 & 256 & 3.283 & 45 & 0 & 0 & 0 & 26726.77 & 69.60 & 100.00 \\
44 & 38400 & 0.000000 & 0.00 & 0.0 & 512 & 3.205 & 24 & 0 & 0 & 0 & 27377.22 & 71.29 & 100.00 \\
45 & 38400 & 0.000000 & 0.00 & 0.0 & 900 & 3.153 & 15 & 0 & 0 & 0 & 27828.73 & 72.47 & 100.00 \\
\hline
\end{tabular}%
}
\begin{tablenotes}
\small
\item TX: Transmissions; ReTX: Retransmissions; REJ: Rejections; TO: Timeouts; R: Data Rate; 
\item S: Measured Efficiency; $S_{Theory}$: Theoretical Efficiency
\item File size: 10968 bytes for all tests
\item The theoretical efficiency accounts for frame overhead, error rates, and propagation delays
\end{tablenotes}
\end{threeparttable}
\end{table}


\begin{figure}[H]
  \centering
  \begin{minipage}[b]{0.45\textwidth}
    \centering
    \includegraphics[width=\textwidth]{../../results/Efficiency vs Frame Size.png}
    \caption{Efficiency as function of frame size}
    \label{fig:efficiency-frame-size}
  \end{minipage}
  \hfill
  \begin{minipage}[b]{0.45\textwidth}
    \centering
    \includegraphics[width=\textwidth]{../../results/Efficiency vs FER .png}
    \caption{Efficiency as function of \ac{FER}}
    \label{fig:efficiency-fer}
  \end{minipage}
  \hfill
  \begin{minipage}[b]{0.45\textwidth}
    \centering
    \includegraphics[width=\textwidth]{../../results/Efficiency vs TProp.png}
    \caption{Efficiency as function of propagation delay ($T_{Prop}$)}
    \label{fig:efficiency-tprop}
  \end{minipage}
\end{figure}

\section*{Conclusions}
This short report provides an efficiency analysis of the implemented protocol. 
After analyzing the results, we conclude that efficiency is significantly influenced by frame size and error rate. 
Larger frame sizes generally yield higher efficiency due to reduced relative overhead, but are more sensitive to higher \ac{BER}, which may lead to more retransmissions and lower efficiency in harsh conditions.
The measured efficiency deviates from theoretical predictions, due mainly to the implementation overheads such as frame header, timeouts, byte stuffing, and acknowledgment delays.
Optimal parameter choices for maximizing throughput include using larger frames and minimizing error rates through appropriate channel conditions.

\vfill
% end of file
